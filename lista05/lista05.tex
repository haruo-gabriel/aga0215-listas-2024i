\documentclass{article}

\usepackage[margin=1cm]{geometry}
\setlength{\parindent}{0pt}
\usepackage{amsmath}
\usepackage{siunitx}

\author{Gabriel Haruo Hanai Takeuchi - NUSP: 13671636}
\title{AGA0215 - Lista 5}
\date{}

\begin{document}

\maketitle

\begin{enumerate}
  \item A órbita da Terra não mudaria, pois a gravidade leva em conta apenas a distância entre os corpos e suas massas. Como nem as distâncias e nem as massas se alteraram, a órbita da Terra não seria afetada.
  \item O Limite de Chandrasekhar de 1,44 massas solares é o limite de massa que uma anã branca consegue suportar gravitacionalmente em um sistema binário cerrado. Após ultrapassar esse Limite, a anã branca colapsa.
  \item O limite de Tolman-Oppenheimer-Volkoff de aproximadamente 3 massas solares é o limite de massa que uma estrela de nêutrons consegue suportar antes de colapsar em um buraco negro, ou seja, o limite que a pressão de nêutrons degenerados consegue suportar.
  \item A dificuldade vem dos gases e poeiras entre o observador e o objeto a ser observado. Esses gases e poeiras dispersam e absorvem a luz, atrapalhando o mapeamento.
  \item As estrelas do disco galáctico se movem de maneira ordenada, em órbitas elípticas ao redor do centro galáctico seguindo o plano do disco galáctico.
  \item As estrelas do halo galáctico se movem aleatoriamente, em órbitas aleatórias ao redor do centro galáctico, mas não necessariamente seguindo o plano do disco galáctico.
  \item As galáxias Sa possuem muito mais gás e poeira interestelar em comparação às galáxias E7. Logo, as galáxias Sa têm uma taxa de formação estelar muito maior que as galáxias E7.
  \item Teorema do Virial.
  \item A curva de rotação da Via Láctea não indica movimento kepleriano, pois a velocidade de rotação não diminui com o aumento da distância ao centro da galáxia.
  \item Os resultados observacionais de cada tipo são diferentes entre si pois, dependendo do ângulo de observação, partes diferentes do núcleo ativo são vistas, resultando em classificações distintas.
\end{enumerate}

\end{document}
