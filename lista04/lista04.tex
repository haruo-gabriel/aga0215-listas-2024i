\documentclass{article}

\usepackage[margin=1cm]{geometry}
\setlength{\parindent}{0pt}
\usepackage{amsmath}
\usepackage{siunitx}

\author{Gabriel Haruo Hanai Takeuchi - NUSP: 13671636}
\title{AGA0215 - Lista 4}
\date{}

\begin{document}

\maketitle

% begin numbered list
\begin{enumerate}
  \item Magnitude aparente é uma escala de luminosidade, em que um corpo celeste é observada a partir da Terra. Magnitude absoluta também é uma escala de luminosidade, mas independe da distância entre o corpo celeste e a Terra.
  \item Os eixos de um diagrama HR são: temperatura e luminosidade.
  \item As duas classificações mais usadas são: espectral e de luminosidade.
  \item Paralaxe espectroscópia é uma técnica para medir a distância de uma estrela a partir de seu espectro ou cor.
  \item O tempo de vida na sequência principal de uma estrela é inversamente proporcional à sua massa. Quanto maior sua massa, menor seu tempo de vida.
  \item O espectro de uma nebulosa de emissão é caracterizado por linhas de emissão. O espectro de uma estrela é caracterizado por linhas de absorção.
  \item O tempo de formação de uma estrela é inversamente proporcional à sua massa. Quanto maior sua massa, menor seu tempo de formação.
  \item A entrada de uma estrela na sequência principal é caracterizada pelo início da fusão nuclear de hidrogênio.
  \item Uma anã marrom é caracterizada por não ter massa e temperatura interna suficientes para iniciar a fusão nuclear de hidrogênio.
  \item A radiação de 21cm é útil para mapear nuvens de gás hidrogênio neutro e coletar informações sobre sua densidade, temperatura, movimentos internos.
  \item O ciclo próton-próton é o processo físico de conversão de hidrogênio em hélio, que ocorrem no núcleo de estrelas de baixa e média massas.
  \item O ciclo triplo-alfa é o processo físico de síntese de elementos cada vez mais pesados a partir do hidrogênio, que ocorrem no núcleo de estrelas de alta massa.
  \item No estágio de gigante vermelha, a temperatura efetiva do Sol diminuirá, enquanto sua luminosidade aumentará.
  \item Uma nebulosa planetária é uma nuvem de gás e poeira que é expelida por uma estrela gigante vermelha no final de sua vida.
  \item Ao final da evolução de uma estrela de massa baixa ou intermediária, resta uma anã branca.
  \item A evolução de uma estrela de alta massa isolada termina com uma supernova.
  \item Supernovas do tipo II e tipo Ia possuem diferentes curvas de luz; O espectro de uma supernova tipo II têm linhas fortes de hidrogênio, enquanto o espectro de uma supernova tipo I não têm linhas de hidrogênio.
  \item Metais preciosos são sintetizados na fase de explosão de uma supernova.
  \item Elementos químicos muito pesados são produzidos nos processos-R, que ocorrem durante a explosão de supernovas do tipo II, em processos de captura de nêutrons.
  \item Uma nova é uma explosão de uma anã-branca em um sistema duplo, com transferência de massa. Uma supernova é a explosão de uma estrela de alta massa isolada ou dupla. Uma supernova é muito mais brilhante que uma nova.
\end{enumerate}

\end{document}