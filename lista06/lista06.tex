\documentclass{article}

\usepackage[margin=1cm]{geometry}
\setlength{\parindent}{0pt}
\usepackage{amsmath}
\usepackage{siunitx}

\author{Gabriel Haruo Hanai Takeuchi - NUSP: 13671636}
\title{AGA0215 - Lista 6}
\date{}

\begin{document}

\maketitle

\begin{enumerate}
  \item A afirmação é falsa. Devido à expansão do Universo, a distância aumentou, sendo maior que 12,1 bilhões de anos-luz.
  \item As relações de Tully-Fischer e Faber-Jackson são indicadores de distância entre galáxias muito distantes. A relação de Tully-Fischer relaciona, para galáxias espirais, a sua velocidade de rotação máxima com a sua luminosidade. A relação de Faber-Jackson relaciona, para galáxias elípticas, a sua dispersão de velocidades com a sua luminosidade.
  \item O efeito de lente gravitacional é causado por objetos massivos que curvam o espaço-tempo ao seu redor. Dessa forma, a luz de outros objetos atrás de um massivo é defletida.
  \item A afirmação é verdadeira quando consideramos o universo em grande escala. As ''enormes diferenças de densidade que existem por exemplo nos núcleos das estrelas comparados com o meio interestelar, ou das galáxias comparadas ao meio intergaláctico'' não são suficientemente relevantes para afetar o princípio cosmológico.
  \item O Fator de Escala $R(t)$ é uma função que considera a expansão/contração do Universo no cálculo de distâncias.
  \item Enquanto o ''Universo observável'' é a porção de todo o Universo que podemos observar, ou seja, que a luz emitida por objetos nessa região teve tempo suficiente para chegar até nós, o ''Universo'' é toda a extensão do espaço-tempo.
  \item O ''confinamento dos quarks'' é uma propriedade de quarks ficarem confinados, devido à forte força nuclear, dentro de hádrons (como prótons e nêutrons). Essa propriedade ocorreu, após o Big Bang, durante o momento de bariogênese.
  \item A energia escura foi descoberta por observações feitas em supernovas do tipo Ia, e explica a aceleração da expansão do Universo.
  \item Essa fase é denominada "inflação cósmica". No Universo atual, as consequências observacionais são a homogeneidade e isotropia do Universo em grande escala.
  \item Os fótons que compõem a radiação cósmica de fundo têm sua origem logo após o Big Bang, e está associada à densidade do Universo.
\end{enumerate}

\end{document}
